\documentclass[article,type=msc,colorback,accentcolor=tud7b]{tudthesis}

%Markus Zopf

\usepackage[english]{babel}
\usepackage{listings}
\usepackage[
    backend=biber, % biber ist das Standard-Backend für Biblatex. Für die Abwärtskompatibilität kann hier auch bibtex oder bibtex8 gewählt werden (siehe biblatex-Dokumentation)
    style=authortitle, %numeric, authortitle, alphabetic etc.
    autocite=footnote, % Stil, der mit \autocite verwendet wird
    sorting=nty, % Sortierung: nty = name title year, nyt = name year title u.a.
    sortcase=false,
    url=false,
    hyperref=auto,
]{biblatex}

\addbibresource{bibliography.bib}

\newcommand{\getmydate}{%
  \ifcase\month%
    \or Januar\or Februar\or M\"arz%
    \or April\or Mai\or Juni\or Juli%
    \or August\or September\or Oktober%
    \or November\or Dezember%
  \fi\ \number\year%
}

\begin{document}
  \thesistitle{Sentiment classification of chess annotations}{}
  \author{Florian Beck}
  \referee{Prof. Dr. Johannes Fürnkranz}{}
  \department{Fachbereich Informatik}
  \group{Knowledge Engineering Group}
  \dateofexam{\today}{\today}
  %\tuprints{12345}{1234}
  \makethesistitle
  \affidavit{Florian Beck}
  
  \input{abstract}
  %auch in deutsch
  \clearpage
  
  \setcounter{tocdepth}{3}
  \tableofcontents
  \setcounter{page}{3}
  \clearpage
  
  \paragraph{List of Figures}
  %\listoffigures\addcontentsline{toc}{chapter}{\listfigurename}
  \clearpage
  
  \section{Introduction}

  \subsection{Motivation}

  \subsection{Problem description}

  \subsection{Goal of the thesis}
    - is it possible to "convert" a chess annotation comment to the appropiate symbol by using a classifier

  \subsection{Structure of the thesis}
  \clearpage
  
  \section{Basics}
    Sentiment Analysis and classification, Ordinal Classification, Word Embeddings, TF-IDF
    mindestens 5 Seiten
  \clearpage
  
  \section{Concept}
    - general problem: structured data easier to evaluate than unstructured data, possibility to build statistics (how good is a product rated?, political attitude?, player/game statistics in chess -> average count of good/bad moves in a chess game, average count of good/bad moves by a specific player (-> talent scouting?))
    --> EXTRACT KNOWLEDGE OUT OF UNSTRUCTURED INFORMATION
    - general [in this case]: comments [chess annotations] should be converted to symbols (=classes) [chess symbols]
    - step 1: define input and output
      - possible input types (symbol: detection of handwritten letters, word/sentence: detection of language, page/file: detection of author
      - possible output types (letter, language, author)
        - "precision": language = language family|language|dialect -> chess: good move vs. brilliant/good/slightly good move
    - step 2: find a database (or similar source) with sufficient information to extract data from
    - step 3: define conditions that filtered data sets has to fulfill
      - language restriction
      - minimal comment length
      - supervised/unsupervised learning
        - in case of
        
    - step 5: tokenize the text
      - handling of punctuation
    - step 6: token preprocessing
      - remove stopwords
      - lowercase
      - stemming
        
      - how to handle order of classes?
      
      - how to handle differences in class counts?
      
      - how to handle too many attributes? -> attribute selection
  
  \section{Approach}

  \subsection{Data set extraction}
    Nur Weka-Classification oder auch ersten Ansatz der NLTK-Classification mit reinnehmen?

  \subsubsection{PGN-format}
    Structure of PGN-file, comments in PGN-file
    
    \begin{table}
      \centering
      \begin{tabular}{| l | l | l |}
    	\hline
    	Symbol & Meaning & Example \\ \hline
    	x & capture & \\ \hline
    	+ & check & \\ \hline
    	\# & checkmate & \\ \hline
    	0-0 & castling kingside & \\ \hline
    	0-0-0 & castling queenside & \\ \hline
    	= & promotion & \\ \hline
      \end{tabular}
      \caption{Basic chess notations}
	\end{table}
	
	\begin{figure}
	  \centering
	  \includegraphics{images/algebraic_notation.png}
	  \caption{Square names in algebraic notation}
	  % https://en.wikipedia.org/wiki/Algebraic_notation_(chess)#/media/File:SCD_algebraic_notation.svg (18.03.2019, 20:56)
	\end{figure}
	
	\begin{figure}
	  \lstset{commentstyle=\color{blue},morecomment=[s]{\{}{\}},moredelim=[is][\bfseries]{\\textbf\{}{\}},moredelim=[is][\color{red}]{\\nag\{}{\}}}
	  \begin{lstlisting}	  
[Event "Deutschland "]
[Site "?"]
[Date "1995.??.??"]
[Round "?"]
[White "Lutz, Ch"]
[Black "Kramnik, V."]
[Result "0-1"]
[ECO "B33"]
[PlyCount "70"]
[EventDate "1995.??.??"]

\textbf{1. e4} {B33: Sicilian: Pelikan and Sveshnikov Variations} \textbf{1... c5 2. Nf3 Nc6 3.
d4 cxd4 4. Nxd4 Nf6 5. Nc3 e5 6. Ndb5 d6 7. Bg5 a6 8. Na3 b5 9. Nd5 Be7 10.
Bxf6 Bxf6 11. c3 O-O 12. Nc2 Bg5 13. a4 bxa4 14. Rxa4 a5 15. Bc4 Rb8 16. b3 Kh8
17. O-O g6 18. Qe2 Bd7 19. Rfa1 19... Bh6} {last book move} \textbf{20. g3} {
Consolidates f4} (20. Nde3 20... Be6 \nag{$14}) \textbf{20... f5} \nag{$11} \textbf{21. exf5 gxf5 22. b4
22... e4} {Black wins space.} \textbf{23. bxa5 Ne5 24. Rb4 Rxb4 25. cxb4 f4 26. Nd4 e3
27. fxe3} (27. Nxf4 \nag{$2} {doesn't work because of} 27... exf2+ 28. Qxf2 28... Bxf4
\nag{$19}) \textbf{27... f3} {He broke from his leash} (27... fxg3 28. hxg3 Qg5 29. Kh2 Nxc4
30. Nf4 \nag{$19}) \textbf{28. Qa2 f2+ 29. Kg2 Qe8 30. Be2 30... Ng4} {
The pressure on the isolated pawn grows} \textbf{31. Bf3} \nag{$4} (31. Qd2 Qh5 32. Bxg4 Qxg4
33. Nf4 Bxf4 34. exf4 Qh3+ 35. Kxf2 Qxh2+ 36. Ke1 Qxg3+ 37. Kd1 Qg1+ 38. Ke2
Bg4+ 39. Kd3 Qxa1 40. f5 \nag{$19}) \textbf{31... Nxe3+} \nag{$19} \textbf{32. Nxe3 Qxe3 33. Qxf2} \nag{$4} {
sad, but how else could White save the game?.} (33. Rd1 Bg7 34. Qb3 Bxd4 35.
Qxe3 Bxe3 36. Be2 \nag{$19}) \textbf{33... Bh3+} \nag{$1} {the final blow} \textbf{34. Kg1} {
Black now must not overlook the idea Re1} (34. Kxh3 {A deflection} 34... Qxf2)
\textbf{34... Qc3 35. Re1 Bd2} (35... Bd2 36. Ne2 36... Qxf3 \nag{$19} (36... Bxe1 \nag{$6} {
is clearly weaker} 37. Nxc3 Bxf2+ 38. Kxf2 \nag{$19}) (36... Rxf3 \nag{$2} 37. Nxc3 Rxf2
38. Kxf2 Bxc3 39. Re7 \nag{$18})) \textbf{0-1}
	  \end{lstlisting}	  

      \caption{Sample PGN game}	
	\end{figure}
    
  \subsubsection{Chessbase-DB}
  	Transformation to PGN, language detection using polyglot
  
  \subsubsection{NAGs}
	Symbols and corresponding NAGs
	
  \subsubsection{Python NLTK}
	RegExp parsing, tokenization, extraction of comment and class  
  
  \subsection{Prepare data for classification}

  \subsubsection{Feature extraction}
    simple features: count(word), advanced features: tf-idf, bigrams, trigrams

  \subsubsection{Generating training instances}
    structure of arff-file
  \clearpage
  
  \section{Experiment setup}
  
  \subsection{Selecting classes}
    distribution how many instances per class, splitting into several problems: 2 classes (!,?), 6 classes (!!,!,!?,?!,?,??), 2 classes (+,-), 7 classes --> introduction of dictionary
    difference if even or odd number of classes ("neutral class")
    
  \subsection{Tokenizer tuning}
    punctuation, special chess notations (\#ce etc.)
    
  \subsection{NLTK Parameters}
    stopwords, stemming, threshold(hapax), lowercase, bigram, trigram

  \subsection{Classifiers}
    which classifier to use? --> MCC (x3), OCC, RF, NBM
  \clearpage  
  
  \section{Evaluation of results}
    tables with number of attributes, tables with accuracies, comparison of confusion matrix\\
    each for simple approach, tf-idf, word embedding
  \clearpage  
  
  \section{Conclusion}

  \subsection{Summary}

  \subsection{Outlook}
  \clearpage
  
  \paragraph{References}
  \clearpage
  
    %\begin{figure}
                 %\centering
                 %\includegraphics[clip,width=0.48\textwidth]{TUDreport-fig}
                 %\caption[Lorem ipsum dolor sit amet]{Lorem ipsum dolor sit amet, consectetuer adipiscing elit. Sed vitae ligula. Integer pharetra ornare eros. Phasellus vitae magna eget metus iaculis consectetuer. Lorem ipsum dolor sit amet, consectetuer adipiscing elit.}
             %\end{figure}

\end{document}
